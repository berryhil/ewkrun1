\subsubsection{WZ production}

\label{sss-WZprod}

%short intro
At the LHC, \WZ\ di-bosons are produced from quark-antiquark initial states at 
leading order and quark-gluon initial states at next-to-leading order. %~\cite{PhysRevD.65.094041}
%Figure~\ref{fig:LOdiagrams} shows the LO Feynman diagrams for \WZ production from $q\bar{q}'$ initial states. 
The SM-allowed $s$-channel diagram has a triple boson vertex and is sensitive to 
ATGC.

%FIGURE LO WZ DIAGRAM
% may include 
% figures/sss-inclboson-diboson-wzprod-wz-s-channel.pdf
% figures/sss-inclboson-diboson-wzprod-wz-t-channel.pdf
% figures/sss-inclboson-diboson-wzprod-wz-u-channel.pdf

%\begin{figure}[htbp]
%  \begin{center}
%  \includegraphics[width=0.9\textwidth]{figures/sss-inclboson-diboson-wzprod-wzdiagram.png}
%  \caption{Leading order Feynman diagrams of \WZ production in the dominant 
%  \qqbar\ channel.}
%\label{fig:sss-WZprod-LOdiagrams}
%\end{center}
%\end{figure}

%decay channels


%analysis CMS and ATLAS.
%ATLAS WZ 8 TEV \cite{Aad:2016ett} 
%ATLAS WZ 7 \TeV~\cite{Aad:2012twa}
% - figure from https://atlas.web.cern.ch/Atlas/GROUPS/PHYSICS/PAPERS/STDM-2012-09/
%CMS WZ at 7+8 \TeV (CMS-PAS-SMP-12-006, to be published)
The ATLAS experiment measured the \WZ\ production cross section in the fully 
leptonic decay channel \ll\lnu\; at $\rts = 7\TeV$~\cite{Aad:2012twa} and $\rts = 8\TeV$~\cite{Aad:2016ett} 
and set limits on charged ATGC.
%Theoretical calculations
% TBD
% WZ->lllv
%Selections
In these analyses the final states involving electrons or muons are considered as signal,
whereas boson decays to taus are considered as background.  
%Backgrounds
The dominant background sources are \Zboson+jet and \ZZ production, accounting for about 40\% of 
the overall background. The overall signal to background ratio is about $4$.
%systematics
The leading systematic uncertainty is related to the data-driven estimation method of the   
background.%, with the dominant Z + jets contributing ($\pm3.8\%$).

% Results at the end
% xsec
The fiducial cross section for the 7 TeV (8 TeV) analysis is defined by $\pt^{\mu,e} > 15\GeV$ for the leptons from the \Zboson\ 
 decay, $\pt^{\mu,e} > 20\GeV$ for the lepton from the \Wboson, $|\eta^{\mu,e}|<2.5$, $\pt^\nu>25\GeV$,
 $|m_ll-m_Z| < 10(25)\GeV$, $M_T^W>20(30)GeV$ and $\Delta R> 0.3$ for the three possible $\ell\ell$ pairings. 
The total cross section requires the mass of the \Zboson\ to be in the range of $66\GeV < |m_Z| < 116\GeV$
to suppress the contribution from $\gamma^*$.
The fiducial and total cross sections are compared to the SM expectation at NLO in Table~\ref{tab:sss-WZprod-xsec}.

\begin{table}[htp]
\begin{center}
\resizebox{\textwidth}{!}{
\begin{tabular}{|c|c|c|c|c|c|}
Experiment & cross section & \rts\;[TeV] & measured  & predicted  & reference  \\ \hline
ATLAS & total & 7  & {19$^{+1.4}_{-1.3}$ (stat.) $\pm{0.9}$ (syst.) $\pm$ 0.4 (lumi.) pb}  & {17.6 $^{+1.1}_{-1.0}$ pb} & \cite{Aad:2012twa} \\
ATLAS & total & 8  & {24.3$\pm 0.6$ (stat.) $\pm{0.6}$ (syst.) $\pm{0.4}$ (theo.) $\pm$ 0.5 (lumi.) pb}  & {21.0  $\pm$ 1.6 pb} & \cite{Aad:2016ett} \\
ATLAS & fiducial & 7  & {92 $^{+7}_{-6}$ (stat.) $\pm{4}$ (syst.) $\pm$ 2 (lumi.) fb}  & -- & \cite{Aad:2012twa} \\
ATLAS & fiducial & 8  & {35 $\pm{0.9}$ (stat.) $\pm{0.8}$ (syst.) $\pm$ 0.8 (lumi.) fb}  & {30$\pm{0.5}$ (PDF) $\pm{0.8}$ (scale) fb} & \cite{Aad:2016ett} \\
\end{tabular}
}
\caption{Summary of measured fiducial and total $\WZ$ production cross sections from ATLAS 
at 7 TeV center-of-mass energies in the $\ll\lnu$ final state. The fiducial definitions differ from the 7 and 8 TeV analysis. Furthermore, the 7 TeV analysis
quotes the sum for the $e$ and $\mu$ channels, the 8 TeV analysis quotes the combination of channels. Thus, the two fiducial cross sections cannot be 
compared directly.}
\end{center}
\label{tab:sss-WZprod-xsec}
\end{table}%


% unfolded spectra
In Figure~\ref{fig:sss-WZprod-ptZ-det} the differential cross section measured in bins of 
$\pt^Z$ is shown, compared to SM prediction and a set of anomalous TGC couplings. 
The normalized $\pt^Z$ spectrum is unfolded and compared to the NLO calculation of \mcatnlo in 
Figure~\ref{fig:sss-WZprod-ptZ-det}, showing good agreement with the SM.


\begin{figure}[htbp]
  \begin{center}
  \includegraphics[width=0.45\textwidth]{figures/sss-inclboson-diboson-wzprod-ptZ-det.pdf}
  \caption{ATLAS measurement of the transverse momentum of the \Zboson in \WZ events ($\pt^Z$) compared with the SM prediction at $\rts = 7\TeV$. For illustration, calculations for a set of anomalous couplings values are also shown. The full uncertainty contains statistical and systematic uncertainties.}
\label{fig:sss-WZprod-ptZ}
\end{center}
\end{figure}

\begin{figure}[htbp]
  \begin{center}
  \includegraphics[width=0.45\textwidth]{figures/sss-inclboson-diboson-wzprod-ptZ.pdf}
  \caption{ATLAS measurement of the normalized fiducial cross-sections in bins of $\pt^Z$ compared with the SM prediction at $\rts = 7\TeV$. The full uncertainty contains statistical and systematic uncertainties.}
\label{fig:sss-WZprod-ptZ-det}
\end{center}
\end{figure}


%%%%
% THIS MIGHT GO INTO THE SECTION ON TGC
% aTGC
Limits on the charged ATGC parameters \dkz, \lz\ and \gz\ are extracted from the transverse momentum distribution
of the \Zboson, $\pt^Z$~\cite{Aad:2012twa} or from the transverse mass of the $\Wboson\Zboson$ system~\cite{Aad:2016ett}. 
The ATGC parameter 95\% CL limits using a dipole form factor with a cut-off of $\Lambda=2\TeV$ and without a form factor are quoted 
in Table~\ref{tab:sss-WZprod-ATGC} from the more constraining $8\TeV$ data set analysis~\cite{Aad:2016ett}.

%\begin{table}\centering
%\caption{Expected and observed 95\% CL on 
%\dkz, \lz\ and \gz.}
%\label{tab:sss-WZprod-ATGC}
%\begin{tabular}{ccccc}
%\hline
%& $\rts$,  & Observed & Observed & Expected \\
%& & $\Lambda=2$~TeV & no form factor & no form factor\\
%\hline
%$7\TeV$ & $\gz$ & $[-0.074, 0.133]$ & $[-0.057, 0.093]$ & $[-0.046, 0.080]$ \\
%$7\TeV$ & $\dkz$ & $[-0.42, 0.69]$ & $[-0.37, 0.57]$ & $[-0.33, 0.47]$ \\
%$7\TeV$ & $\lz$ & $[-0.064, 0.066]$ & $[-0.046, 0.047]$ & $[-0.041, 0.040]$ \\
%\hline
%\end{tabular}
%\end{table}


\begin{table}\centering 
\begin{tabular}{cccc}
\hline
 $\Lambda$ & Coupling & Expected  & Observed \\
\hline
2 TeV 	& $\gz$ 		& [$-0.023$, $0.055$]  & [ $-0.029$, $0.050$] \\
2 TeV 	& $\dkz$ 	& [$-0.17$, $0.25$]    & [$-0.19$, $0.30$] \\
2 TeV 	& $\lz$ 	& [$-0.016$, $0.016$]  & [$-0.016$, $0.016$] \\
\hline
\end{tabular}
\caption{Expected and observed 95\% CL on \dkz, \lz\ and \gz\; for a cut-off parameter $\Lambda = 2\TeV$ and $\Lambda = \infty$.}
 \label{tab:sss-WZprod-ATGC}
\end{table}
 

% WZ->llqq ????









