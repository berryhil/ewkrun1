
We confine discusion of results in this review to experimental tests
at the LHC which directly and uniquely test the electroweak
interactions of the known standard model particles, excluding the
Higgs boson. A simple but general framework to study those
interactions is through effective field theory extensions of the
standard model which respect the standard model $SU_C(3) \otimes
SU_L(2) \otimes U_{EM}(1)$
symmetry~\cite{Weinberg:1978kz,Weinberg:1980wa,Georgi:1994qn}.  These
extensions can be generally written in an operator product expansion
(OPE)

$$ \mathcal{L} = \mathcal{L}_{SM} + \sum_i \frac{c_i}{\Lambda^{2n}}{\cal O}_i + \cdots$$

where $c_i$ are dimensionless coefficients, $\Lambda$ is some energy
scale corresponding to the scale of new physics, and ${\cal O}_i$ are
operators constructed from Standard Model fields.  In principle, given
an ultraviolet-complete theory of new physics with possibly new
particle content, the $c_i$ at the electroweak scale are calculable
and can be compared with the observed experimental footprint. In the
context of electroweak studies, especially for the study of
gauge-boson self-interactions, this is often restricted to $CP$-even
operators of mass dimension six (the
lowest allowable).  In that case, there are only three independent
operators with three independently measurable Wilson coefficients. The
precise definition of the independent coefficents depends upon the
basis chosen to express the operators.  The most common contemporary
convention is the LEP basis, for which the coefficients are labeled
$c_{WWW}$, $c_W$, and $c_B$~\cite{Hagiwara:1993ck,Degrande:2012wf},
which correspond to combinations of $WW\gamma$ and $WWZ$ interactions.
Prior to the most recent Run 1 results, these same operators (up to a
simple linear combination) were constrained via dimensionless
couplings, known as $\lambda$, $\kappa$, and $g_1$; these can be
easily translated into the Wilson coefficients $c_i$ by a linear
transformation~\cite{Degrande:2012wf}.  By measuring total or
differential cross sections of processes which isolate triple gauge
boson self-interactions (TGCs), predictions from the EFT can be
compared to the data and allow for bounding the values of the
coefficents, typically expressed as the dimensionful ratio
$c_i/\Lambda^2$.

In understanding all of the conceivable quartic gauge boson
interactions (QGCs), it is useful to also consider EFTs for which
dimension 6 operators are suppressed, and the leading order OPE has
terms of mass dimension 8, with dimensionful coefficients
$c_i/\Lambda^4$.  Dimension 8 theories have up to 19 independent gauge
boson self-interaction operators, including the following quartic
combinations: $WWWW$, $WWZZ$, $WWZ\gamma$, $WW\gamma\gamma$, as well
as non-SM interactions $ZZZZ$, $ZZZ\gamma$, $ZZ\gamma\gamma$, and
$Z\gamma\gamma\gamma$~\cite{Eboli:2006wa,Degrande:2013kka}.

Not all $c_i/\Lambda^n$ give rise to unitary or perturbatively
calculable field thoeries, therefore care must be taken to specify,
for each experimental observable, which range is applicable.
Alternatively, some (higher mass dimension) unitarization procedure
can be adopted: an energy cutoff on the cross section, a form factor
which moderates the high-energy behavior (usually of the form
$(1+q^2/\Lambda^2)^{-2}$), or a $K$-matrix unitarization
procedure~\cite{Alboteanu:2008my,Chung:1995dx}.  In general, the Run 1
data on dimension-6 EFTs do not explore the range of coefficients and
energies which would escape unitarity bounds, in which case the bare
results have a straightforward interpretation.  It is typical,
however, for dimension-8 EFTs to violate unitarity in the Run 1 LHC
energies and sensitivities, so the details of unitarity and
unitarization are critical in presenting and interpreting those
results.

Another framework for testing the electroweak theory is through global
fits of the SM electroweak parameters to the high-precision
electroweak data available in $W$ and $Z$ production.  Poor
goodness-of-fit for one or more precision observables is indicative of
new electroweak interactions at the loop level contributing to
electroweak boson production and decay; these constraints can also be
translated into bounds on EFT coefficients~\cite{Falkowski:2014tna}.
The most recent global fits~\cite{Baak:2014ora,Ciuchini:2013pca}
include, among others, the LEP and SLD precision data in $Z$
production, the Tevatron measurement of the $W$ mass, and the
high-precision measurements of the top quark and Higgs boson masses at
the LHC.  These fits generally predict values for the $W$ mass and
weak-mixing angle $\sin^2\theta^{eff}_{W}$ more accurate than have
been directly obtained, providing an opportunity at the LHC to
over-constrain the global fit further.

Improving upon either of these frameworks for electroweak model
parameters at the LHC is only possible with state-of-the-art
understanding of perturbative QCD, parton distribution functions
(PDF), and electroweak radiative corrections.  In general, for Run 1
LHC results multiboson production has been studied with
next-to-leading order (NLO) QCD matrix elements combined with parton
showering to provide a particle-level simulation and some emulation of
higher-order radiative and non-perturbative corrections. The PDFs and
underlying event modelling typically include the LHC data themselves.
For precision single boson production next-to-next-to-leading order
(NNLO) calculations were used, along with NLO electroweak radiative
corrections.  The LHC data typically improve upon contemporary PDFs in
this domain, motivating precision measurement of boson production to
constrain them further.  In the future, multiboson and vector-boson
scattering studies will require NNLO QCD and NLO electroweak matrix
elements, parton-shower matching at that order, and PDFs consistently
provided at that order including all of the Run 1 LHC data, as the Run
1 data are already precise enough to be sensitive to these effects in
some areas~\cite{Badger:2016bpw}.
