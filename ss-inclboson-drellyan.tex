At a hadron collider, the most fundamental tests of electroweak boson
couplings to fermions are measurements of the kinematic properties of
Drell-Yan (DY) lepton pair production.  At leading order, Drell-Yan
production occurs when a quark--anti-quark pair in the intial state
annihilates into an electroweak boson, which subsequently decays to a
lepton pair. Differential cross section calculations exist for
next-to-next-to leading order (NNLO) QCD corrections as well as NLO
electroweak corrections. In the EFT context, such a process is
sensitive to four-fermion contact interactions of the type

\begin{equation}\label{lagrangian}
\begin{array}{r@{\,}c@{}c@{\,}l@{\,}l}
\mathcal L = \frac{g^2}{\Lambda^2}\;[ && \eta_{\rm LL}&\, (\overline q_{\rm L}\gamma_{\mu} q_{\rm L})\,(\overline\ell_{\rm L}\gamma^{\mu}\ell_{\rm L}) \nonumber \\
& +&\eta_{\rm RR}& (\overline q_{\rm R}\gamma_{\mu} q_{\rm R}) \,(\overline\ell_{\rm R}\gamma^{\mu}\ell_{\rm R}) \\
&+&\eta_{\rm LR}& (\overline q_{\rm L}\gamma_{\mu} q_{\rm L}) \,(\overline\ell_{\rm R}\gamma^{\mu}\ell_{\rm R}) \\
&+&\eta_{\rm RL}& (\overline q_{\rm R}\gamma_{\mu} q_{\rm R}) \,(\overline\ell_{\rm L}\gamma^{\mu}\ell_{\rm L})& ] \: ,\nonumber
\end{array}
\end{equation}
where $g$ is a coupling constant, $\Lambda$ is the contact interaction scale,
and $q_{\rm L,R}$ and $\ell_{\rm L,R}$ are left-handed and right-handed quark and
lepton fields, respectively. The parameters $\eta_{i,j}$ denote the relative interference of the operators;
the experiments have considered the cases $\eta_{\rm LR} = \eta_{\rm RL} = \pm 1$,
$\eta_{\rm LL} = \pm 1$, or $\eta_{\rm RR} = \pm 1$.

Experiments select electron or muon pairs above trigger thresholds:
CMS selects leading lepton $\pt >$ 17 GeV and second leading lepton
$\pt >$ 8 GeV inclusively, and ATLAS selects high mass events with
both lepton $\pt >$ 25 GeV.  Backgrounds to Drell-Yan production are
relatively small, and consist of real prompt lepton pair production
from top quark or boson pairs, as well as fake electrons from QCD
jets.  The real lepton pair background is flavor democratic, and can
therefore be reliably estimated from $e\mu$ pair production.  Fake
electron production is typically estimated from background enriched
QCD jet samples, from which the fake electron rate can be measured,
convolved with electron-jet control samples.

Figure~\ref{fig:ss-inclboson-drellyan-atlas7tev} shows the Drell-Yan
cross section at high electron pair mass measured by ATLAS at 7
TeV~\cite{Aad:2013iua}.  The cross section uncertainty is
predominantly systematic below 400 GeV in pair mass and predominantly
statistical above 400 GeV.  The data are compared with an NNLO QCD
prediction with NLO electroweak corrections, provided by the
\texttt{FEWZ} 3.1
generator~\cite{Melnikov:2006kv,Gavin:2010az,Li:2012wna}.  The
prediction also includes photon induced lepton pair production, which
generally increases cross section estimates by a few percent. The
\texttt{FEWZ} prediction generally underestimates the cross section,
however a correlated chi-squared analysis concludes that this is not
statistically significant.

Figure~\ref{fig:ss-inclboson-drellyan-cms8tev} shows the Drell-Yan
cross section for electron or muon pairs measured by CMS at 8
TeV~\cite{CMS:2014jea}.  Agreement with the \texttt{FEWZ} prediction
is observed over the entire measured mass range, from 15 GeV to 2000
GeV.  CMS has also measured the double differential cross section with
respect to dilepton rapidity in several bins of dilepton mass, as well
as a differential cross section ratio between the 8 TeV and 7 TeV
data, which has small experimental and theoretical uncertainties.

In the absence of observed disagreements with predictions at the
highest dilepton masses, the data are analyzed to constrain the size
of anomalous contact interactions. Assuming a fixed, strong value for
the coupling ($g^2/4\pi = 1$), limits can be obtained on the contact
interaction scale $\Lambda$.  ATLAS estimates a lower limit of 17 to
26 TeV on $\Lambda$, where the strongest lower limits correspond to
constructive interference scenarios (especially LR+RL), and the
weakest to destructive interference scenarios~\cite{Aad:2014wca}.  CMS
has limits with similar sensitivity estimated for LL contact
interactions~\cite{Khachatryan:2014fba}.

%ATLAS low-mass Drell-Yan $7 \TeV$~\cite{Aad:2014qja}
%ATLAS Z PT $7 \TeV$~\cite{Aad:2014xaa}
%ATLAS Z phistar $7 \TeV$~\cite{Aad:2012wfa}
%CMS Drell--Yan $7 \TeV$~\cite{Chatrchyan:2013tia}
%CMS angular coefficients $8 \TeV$~\cite{Khachatryan:2015paa}
%CMS Z PT and rapidity $8 \TeV$~\cite{Khachatryan:2015oaa}
%CMS dilepton contact interactions~\cite{Khachatryan:2014fba}
%ATLAS dilepton contact interactions~\cite{Aad:2014wca}

\begin{figure}[p]
    \centering
    \includegraphics[height=0.3\textheight]{figures/ss-inclboson-drellyan-atlas7tev}
    \caption{Measured differential cross-section at the Born level within the
    fiducial region (electron $\pt > 25 \GeV$ and $|\eta| < 2.5$) with statistical,
     systematic, and combined statistical and systematic (total) uncertainties,
     excluding the 1.8\% uncertainty on the luminosity.
      On the left, in the upper ratio plot, the photon-induced (PI)
     corrections have been added to the predictions obtained from the MSTW2008,
     HERAPDF1.5, CT10, ABM11 and NNPDF2.3 NNLO PDFs, and for the MSTW2008 prediction
     the total uncertainty band arising from the PDF, $\alpha_s$, renormalisation
     and factorisation scale, and photon-induced uncertainties is drawn. The lower
     ratio plot shows the influence of the photon-induced corrections on the
     MSTW2008 prediction, the uncertainty band including only the PDF, $\alpha_s$
     and scale uncertainties.}
    \label{fig:ss-inclboson-drellyan-atlas7tev}
\end{figure}

\begin{figure}[p]
    \centering
    \includegraphics[height=0.3\textheight]{figures/ss-inclboson-drellyan-cms8tev}
    \caption{The DY differential cross section as measured in the combined
dilepton channel and as predicted by NNLO \texttt{FEWZ} 3.1 with CT10 PDF
calculations, for the full phase space.}
    \label{fig:ss-inclboson-drellyan-cms8tev}
\end{figure}
