\subsubsection{WW production}
\label{sss-WWprod}

%short intro
The \WW\ production process has the highest production cross section
among the massive vector diboson processes. It is also an important
background process to Higgs production and to searches for new physics.
%analysis CMS and ATLAS.
%    ** ATLAS (5.6ifb, 7TeV, TGC), Phys. Rev. D 87, 112001 (2013), http://arxiv.org/abs/1210.2979
%	    ATLAS (20ifb 8TeV), https://atlas.web.cern.ch/Atlas/GROUPS/PHYSICS/CONFNOTES/ATLAS-CONF-2014-033/ 
%    ** CMS (7 TeV, 5 fb-1, TGC)  WW cross section in the lvlv channel at 7 TeV  http://arxiv.org/abs/1306.1126
%    ** CMS (8 TeV, 3.5 fb-1 WW, 5.3 fb-1 ZZ) , Phys. Lett. B 721 (2013) 190, WW and ZZ at 8 TeV, http://arxiv.org/abs/1301.4698 
%	    CMS (20ifb, 8TeV), submitted to EPJC,  http://arxiv.org/abs/1507.03268
ATLAS and CMS, observed the \WW\ production process in 
the fully leptonic channel and published results for 7 TeV 
(ATLAS~\cite{ATLAS:2012mec},CMS~\cite{Chatrchyan:2013yaa}) and
8 TeV (CMS~\cite{Chatrchyan:2013oev}) centre-of-mass energy. 
%decay channels
Three final states, namely $ee$, $\mu\mu$, and $e\mu$ are included in the analyses. 
The contribution from leptonically decaying $\tau$ leptons is included in the signal
definition. Although the production cross section is relatively high, the signature of two opposite 
sign leptons and missing transverse energy is shared with many processes and a careful
control of the backgrounds is necessary to achieve a precise measurement.

%Theoretical calculations
% TBD -> General section about theoretical calculations?

%Selections
Candidate $\WW$ events are selected by by requiring two oppositely charged leptons 
accompanied with large \MET. 
%Backgrounds
The dominant background sources are \ttbar\; and single top quark, 
$\W/$+jets, followed by $\Zzero/\gamma^{*}$+jets production.
To suppress the dominant \ttbar\; background, events with one or more jets are rejected.   
Additional requirements on \MET\ and the use of top quark-taggers further reduce the residual background
%ATLAS: 685 WW, 275 background => b / (s+b) = 29%
%CMS: 824WW, 369 background => b / (s+b) = 31%
to about 30\%.  
%systematics
The dominant systematic uncertainty is related to the jet veto efficiency 
and estimated to about 5\% for the \WW\, production. 
The experiments quote a theoretical uncertainties on the signal acceptance due to 
variations of the parton distribution functions and renormalisation and factorization 
scale in the range of 1-2\%.

% Results at the end
% total xsec
% CMS 7TeV: s(ww) = 52.4 +- 2.0 (stat) +- 4.5 (syst) +- 1.2 (lum) pb
% ATLAS 7TeV:
% CMS 8TeV:
Both ATLAS and CMS provide a measurement of the total cross section for the process $pp \rightarrow \WW$.
ATLAS 

% fiducial xsec

The fiducial cross section 

% unfolded spectra


%%%%
% THIS MIGHT GO INTO THE SECTION ON TGC
% aTGC









