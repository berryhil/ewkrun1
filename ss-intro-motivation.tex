\label{ss-intro-motivation}

%Physics motivation
With the discovery of the Higgs boson in 2012~\cite{Chatrchyan201230,
Aad20121} the standard model of particle physics seemed complete, but
fundamental questions remain to be answered, including the
constituents of dark matter, the relative abundance of matter over
anti-matter, and the unification of forces at the Planck scale.

The standard model of the electroweak interactions is a quantum field
theory unifying the electromagnetic and the weak force into one gauge
theory transforming under the gauge group $SU_L(2) \otimes U_{EM}(1)$.
The electroweak Lagrangian can be represented as

$$ \mathcal{L}_{EW} = \mathcal{L}_{boson} + \mathcal{L}_{fermion} + \mathcal{L}_{higgs} + \mathcal{L}_{yukawa} $$
%\begin{equation}
%\end{equation}

with $\mathcal{L}_{boson}$ the kinetic and self-interaction term of
bosons, $\mathcal{L}_{fermion}$ the kinetic term describing the
interacion of fermions with bosons, $\mathcal{L}_{higgs}$ the Higgs
term which generates the gauge boson masses and coupling to the Higgs
field, and the Yukawa term $\mathcal{L}_{yukawa}$ describing the
interaction between the fermion and Higgs field.
%EWK precision measurements
Precision measurements in the electroweak sector allow to test the
plausibility of various extensions to the standard model
Lagrangian. In the framework of anomalous gauge couplings and its
reformulation as effective field theory, the standard model can be
extended with additional generic terms in the Lagrangian to describe
new physics. The measurement of electroweak production processes
allows to verify the validity of the standard model in this framework
at the high collision energies the Large Hadron Collider (LHC) is
providing.

%LHC and detectors
The LHC was designed to primarily search for the Higgs boson and new
phenomena, but the requirements of proton-proton collisions at high
collision energy and luminosity also enable precision electroweak
measurements. The LHC
% photon structure and benchmark perturbative QCD
provides a rich testing ground to benchmark perturbative QCD over a
wide range of scales, and to determine the structure of the proton.
Both are essential ingredients to make precise electroweak
measurements, as higher order corrections in perturbative QCD have a
substantial effect on the theoretical predictions to standard model
processes, and the knowledge of the proton structure becomes a major
uncertainty for processes with very high partonic center-of-mass,
i.e. the region where new physics is expected to become visible.
% Quantitative statement possible?

%LHC machine
The LHC collides protons on protons in a circular ring of 27\;km
circumference with an energy of up to $7\TeV$ per beam, corresponding
to a collision energy at the center-of-mass of $\rts=14\TeV$, and a
design luminosity of $10^{34}\cm^{-2}\s^{-1}$.
% >> put this into perspective by comparing directly to tevatron and LEP parameters?
%Detectors
The two general purpose detectors at the LHC, ATLAS and CMS, are able
to record a wide range of physics process. They differ in the
technical details of detector layout and data-acquisition design but
have common design goals, namely good detector coverage for charged
and neutral particles, excellent resolution to measure the positions
and momenta of charged particles, and precision calorimetry. The other
two main experiments at the LHC are optimized for recording collision
of heavy ions in the case of ALICE or for precision measurements of
b-hadrons and CP violation (LHC-b).

% data sets
In the run period from 2010 to 2011 about $6\ifb$ of integrated
luminosity at a center-of-mass energy $\rts=7\TeV$ were delivered. In
2012 the energy was raised to $\rts=8\TeV$ and the integrated
luminosity produced was $23\ifb$.  After a long shutdown LHC resumed
running in 2015 at an increased energy of $\rts=13\TeV$, and was able
to provide $4.5\;\ifb$ of collision data.  The experiments ATLAS and
CMS have published results on all three data sets, but the majority of
electroweak results available up to now are based mainly on the 7~TeV
data set and to some extend on the 8~TeV data sets. Generally, the
precision measurements of cross section and differential distributions
involve more analysis effort compared to e.g. searches for new
resonances, hence the different time-scale for this category of
measurements.

% electroweak objects produced in run-1

% W,Z produced, di-boson final states? 
% MW mass, prospects vs reality?

% mention typical systematic uncertainties on leptons, jets, met

% mention ewk parameters from LHC in PDG?


%outline
In the following we will give a brief review
%the electroweak physics programme at the LHC (section~\ref{ss-lhc-physics}), followed by an overview of 
of recent theory developments (section~\ref{s-theory-overview}), and a
comprehensive summary of the electroweak results from ATLAS and
CMS. In section~\ref{s-inclboson} we summarise the results on
inclusive vector boson production (Drell-Yan process, inclusive
di-boson and tri-boson production), and continue in
section~\ref{s-exclboson} with exclusive boson production in the
vector boson fusion and vector boson scattering processes. Finally,
the LHC results on electroweak precision tests are discussed in
section~\ref{s-ewk-prec-tests}.


