%Summary & Outlook
The ATLAS and CMS experiments at the LHC have published a first set of electroweak measurements with the 
run-1 data-sets at a collision energy of $7\TeV$ and $8\TeV$,
%, and some first measurements with run-2 data at $\rts=13\TeV$. 
probing the electroweak sector at the hitherto highest available scales.
%DY
We started with the discussion of precision measurents of Drell-Yan production, a well understood electroweak
process that in turn is used to validate theoretical calculations with higher order corrections in
$\alpha_s$ and the PDF's of the proton. The total and differential cross section measurements help to 
reduce the theoretical cross section uncertainties of multi-boson processes, which are generally of the same order as the current
experimental uncertainties.  
%diboson
We reviewed the set of results available so far on inclusive di-boson production at the LHC. All inclusive di-boson final states
($\WW,\WZ,\ZZ,\Wg,\Zg$) have been observed and cross section measurements are available for collision energies of 7 TeV and 8 TeV. 
The complete set of measurements from both ATLAS and CMS including leptonic and semi-leptonic decay channels
is expected to be published towards the end of 2016.
%WWW+VBS
A highlight of the electroweak physics program at the LHC is the observation of electroweak production channels 
that became accessible for the first time, namely processes with 
three vector bosons in the final state, and vector boson scattering processes that are 
characterised by two forward jets and two vector bosons in the final state.

%aGC limits
For most total and fiducial cross section measurements with the full run-1 dataset the systematic uncertainties 
already limit the precision. The exception are processes with cross sections below a few fb, e.g. tri-boson producion. 
To probe the validity of the standard model at high partonic centre-of-mass energy, $\hat{s}$, the limits
set in the framework of anomalous gauge couplings are surpassing legacy measurements at LEP and the Tevatron,
and are expected to further improve with increases $\rts$ and integrated luminosity. 

%legacy run-1 results
The set of run-1 results once completed will present a unique legacy of electroweak precision results at 
$\rts$ values of 7 and 8 TeV.
%outlook
The run-2 analysis extend the energy range to $\rts=13\TeV$ 
and provide even more precise limits on anomalous gauge couplings. Potentially, 
the combination of ATLAS and CMS results will allow more stringent constraints. Also the combination
with Higgs measurements in the framework of EFT with shared anomalous coupling operators is being 
actively pursued and will allow a further reduction of the allowed parameter space. 
Precision measurements or first observations of more Tri-boson production and VBS processes will become possible with the full 
run-2 data set. 


 
